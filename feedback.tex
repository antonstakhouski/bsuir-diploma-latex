% Содержимое данного документа позаимсвовано из Приложения Е из документа http://www.bsuir.by/m/12_113415_1_66883.pdf

\thispagestyle{empty}

%\begin{singlespace}

{
  \begin{center}
    \begin{minipage}{0.8\textwidth}
      \begin{center}
        {\normalsize ОТЗЫВ}\\
	      руководителя дипломного проекта\\
	      на студента гр. 450502\\
	      Стаховского Антона Владимировича\\
      \end{center}
    \end{minipage}
  \end{center}

На время дипломного проектирования перед дипломником была поставлена задача разработать систему
	функционального контроля технических средств комплекса машин управления артиллерийского дивизиона. Система
	должна выполнять задачи по тестированию подключенных к ЭВМ технических средств, предоставлять средства для
	автоматизации отчетности.

Тема является актуальной, т.\,к. в современных артиллерийских комплексах имеется большое количество различных
	технических средств.
Возможность своевременной диагностики имеющихся в работе устройств неисправностей позволяет избежать потерь личного
	состава, порчи оборудования и потери преимущества на местности.

В ходе дипломного проектирования Стаховским~А.\,В. была разработана система, позволяющая выполнять автоматическое
	тестирование широкого спектра устройств. Система выполняет автоматическое ведение журнала
	тестирования устройств, а также предоставляет средства для просмотра журнала. Разработанная
	система выполняет все поставленные задачи.

В процессе проектирования Стаховский~А.\,В. проявил большую самостоятельность и инициативность.
Работа над проектом велась ритмично и в соответствии с календарным графиком.
Пояснительная записка и графический материал оформлены аккуратно и в соответствии с требованиями ЕСКД.

Стаховский~А.\,В. на основании анализа большого количества специализированной литературы разработал алгоритмы,
	позволяющие получить наиболее полную информацию о состоянии подключенных устройств.


%Приведенные расчеты и программное обеспечение "--- это результат высокоэффективной работы над темой и умения использовать техническую литературу и применять на практике знания, полученные за годы обучения в университете.

%Работа над проектом велась ритмично и в соответствии с календарным графиком.
%Пояснительная записка и графический материал оформлены аккуратно и в соответствии с требованиями ЕСКД.

	Разработанная в ходе дипломного проектирования система была представлена на
	35-й научно-технической конференции молодых специалистов компании.

Результаты, полученные в дипломном проекте, могут быть использованы в разработке комплекса автоматизации комплекса
	машин управления артиллерийского дивизиона.

%Дипломный проект Москаленко~О.\,Н. соответствует техническому заданию и отличается глубокой проработкой темы и выполнен с применением современных прогрессивных технологий.

Считаю, что Стаховский~А.\,В. освоил технику инженерного проектирования систем
	и заслуживает присвоения квалификации <<инженер-\break системотехник>>.

  %\vfill

	\bigskip
	\bigskip
  \noindent
  \begin{minipage}{0.54\textwidth}
    \begin{flushleft}
	    Ведущий инженер-программист \company~ -- управляющая компания
	    холдинга «Геоинформационные системы управления»
    \end{flushleft}
  \end{minipage}
  \begin{minipage}{0.44\textwidth}
    \begin{flushright}
       Т.\,В.~Державская
    \end{flushright}
  \end{minipage}
}

%\end{singlespace}

\clearpage
