\section{ПРОГРАММА И МЕТОДИКА ИСПЫТАНИЙ}
\label{sec:test}
Тестирование разрабатываемых программных продуктов является крайне важным процессом. Тестирование позволяет избежать
лишних затрат на разработку, так как ошибки, возникающие в процессе разработки ПО, вскрываются как можно раньше. Также
исправление как можно большего количества ошибок перед релизом позволяет избежать затрат на исправление ошибок после
внедрения ПО.

Тестирование позволяет не только выявить ошибки в программном коде, но и вычислить фрагменты кода, требующие
оптимизации. Выполнение этой функции облегчают инструменты трассировки и профилирования кода.

Тестирование программы проводилось на компьютере со следующими характеристиками:
\begin{itemize}
	\item центральный процессор -- Intel Core i5-5200U с тактовой частотой 2.20ГГц;
	\item объем оперативной памяти -- 4 ГБ;
	\item операционные системы: ArchLinux x86\_64, Windows 10 x86\_64 Education.
\end{itemize}

Тестирование на ОС Gnu/Linux было проведено в связи с удобством таких инструментов отладки, как Valgrind, которые не
имеют версии для Windows. Для запуска Windows версии приложения под Linux использовался Wine.

\subsection{Инструменты отладки}
В качетве отладчика был выбран GDB, так как он может быть использован для тестирования как и в ОС Linux, так и в ОС
Windows. GDB также позволяет использовать отладчик, используя графический интерфейс. Для этого можно использовать
подключение GDB в качестве инструмента отладки к многим популярным IDE. Особый интерес представляет возможность
использования данного инструмента в Qt Creator, поскольку данный проект активно взаимодействует с библиотеками Qt.

Запуск
\medskip
\begin{minted}{c++}
\end{minted}
\medskip
