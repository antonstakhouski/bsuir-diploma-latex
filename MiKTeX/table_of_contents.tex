% Зачем: Содержание пишется полужирным шрифтом, по центру всеми заглавными буквами
% Почему: Пункт 2.2.7 Требований по оформлению пояснительной записки.
\renewcommand \contentsname {\centerline{\normalfont\normalsize{СОДЕРЖАНИЕ}}}

\makeatletter
\let\oldcontentsline\contentsline
\def\contentsline#1#2{%
  \expandafter\ifx\csname l@#1\endcsname\l@section
    \expandafter\@firstoftwo
  \else
    \expandafter\@secondoftwo
  \fi
  {%
    \oldcontentsline{#1}{\MakeTextUppercase{#2}}%
  }{%
    \oldcontentsline{#1}{#2}%
  }%
}
\makeatother

% Зачем: Не захламлять основной файл
% Примечание: \small\selectfont злостный хак, чтобы уменьшить размер шрифта в ToC
{
\normalsize\selectfont
\tableofcontents
\newpage
}
