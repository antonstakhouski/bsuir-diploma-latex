\sectioncentered*{Система функционального контроля технических средств комплекса машин управления артиллерийского
дивизиона}

Здравствуйте, уважаемые члены ГЭК. Тема моего дипломного проекта -- <<Система функционального контроля технических
средств комплекса машин управления артиллерийского дивизиона>>.

В состав комплекса машин управления огнем входят:
\begin{itemize}
	\item машина управления командира дивизиона;
	\item командно-штабная машина дивизиона;
	\item машина управления командира батареи;
	\item машина управления старшего офицера батареи.
\end{itemize}

В одном дивизионе имеется несколько машин разного уровня управления, содержащих в своем
составе разный набор ТС.
Например, метеокомплект стоит только на нескольких машинах, радиостанции имеются в каждой машине, бесплатформенная
инерциальная навигационная система присутствует также на каждой машине, но имеют разные типы устройства,
ЛВС присутствует в каждой машине.

Во время эксплуатации военной техники внезапный отказ технических средств и локальной вычислительной сети может привести
к серьезным потерям личного состава, порче оборудования, потере преимущества на местности.
В таких условиях автоматизация процессов проведения тестирования является одной из наиболее приоритетных задач.

Целью данного дипломного проекта является разработка и реализация системы автоматизации процессов тестирования
технических средств и каналов обмена данными в локальной сети, упрощение ведения отчетности. Цели и функции системы
представлены на вводном плакате.
Данная система в первую очередь ориентированна на использование артиллерийским дивизионом, но при небольших доработках
программные модули могут быть также использованы в решениях для других армейских подразделений.

Разработанная в ходе дипломного проектировании система функционального контроля технических средств не имеет широко
известных прямых аналогов. Отсутствие конкурирующих систем вызвано тем следующими причинами:
\begin{itemize}
	\item технические средства, предназначенные для нужд армии имеют ограниченное распространение;
	\item протоколы обмена данными между техническими средствами и ЭВМ зачастую являются закрытыми;
	\item модуль функционального контроля зачастую является частью закрытого проприетарного ПО, используемого для
		автоматизации задач личного состава;
	\item разнообразие устройств, протоколов обмена.
\end{itemize}

К косвенным аналогам относятся система автоматизации ЖАТ РЖД и система функционального контроля электронной аппаратуры,
представленные в ПЗ.

Для успешного выполнения поставленной цели, работа над проектом была разбита на следующие задачи:
\begin{itemize}
    \item выбор технологий, удовлетворяющих требованиям;
    \item разработка управляющего модуля;
    \item разработка алгоритмов функционального контроля навигационной системы;
    \item разработка алгоритмов функционального контроля метеостанции;
    \item разработка алгоритмов функционального контроля радиостанции;
    \item разработка алгоритмов функционального контроля прибора наблюдения разведчика <<Капонир>>;
    \item разработка алгоритмов функционального контроля лазерного\break целеуказателя-дальномера;
    \item разработка алгоритмов функционального контроля принтера;
    \item разработка алгоритмов журналирования;
    \item разработка программы для просмотра журнала тестирования.
\end{itemize}

Разработанное в ходе дипломного проектирования программное обеспечение предназначено для развертывания в подвижном
комплексе средств автоматизации управления.
Этот подвижный комплекс средств автоматизации управления, размещенный в подвижном объекте на шасси автомобиля повышенной
грузоподъемности, содержит несколько автоматизированных рабочих места(АРМ) должностных лиц, размещенных в кузове-фургоне
подвижного объекта, оборудованных средствами вычислительной техники и средствами передачи данных,
радиостанцию, локальную вычислительную
сеть, специальный принтер.

Диплом написан на базе предприятия~\company, которое занимается созданием автоматизированных систем
управления войсками в звене фронт -- армия -- дивизия -- полк -- батальон, комплексной интегрированной автоматизированной

системы управления всеми видами и родами войск в этих звеньях.

При разработке системы были использованы библиотеки протоколов устройств и программы-имитаторы, разработанные в компании
~\company. Данные программные модули позволили облегчить процесс разработки.

В процессе разработки система  функционального контроля была разбита на следующие модули и блоки, представленные на
структурной схеме.

Данные являются максимально независимыми. Управляющий модуль осуществляет сканирование подключенных устройств,
предоставляет возможность доступа к программе просмотра журнала. При тестировании подключенных устройств управляющий
модуль передает управление модулю, выполняющего тестирование выбранного устройства. После тестирования выбранных
устройств, данные передаются в блок журналирования для записи в журнал.

Более подробно состав разработанных модулей можно увидеть на диаграмме классов.
Классы с префиксом \texttt{VTest} и \texttt{QTest} используются для тестирования соответствующих устройств.
Классы с префиксом \texttt{QTest} выполняют тестирование с помощью объектов классов стандартной библиотеки Qt.
Классы с префиксом \texttt{VTest} (от Verify) используются для тестирования устройств на основе анализа информационных и
статусных сообщений от приборов.

Классы \texttt{Journal} и \texttt{JournalEntry} отвечают за журналирование.

Классы \texttt{JSON}, \texttt{JSONdevice}, \texttt{JournalViewer}, \texttt{infoWindow} и \texttt{ListFilterDevices}
используются в программе просмотра журнала.

Класс \texttt{OffLineFuncControl} -- реализация управляющего модуля.

Класс \texttt{Imitator\_R181} -- класс, отвечающий за работу программы-имитатора, используемой для тестирования работы
радиостанции путем отправки сформированных пользователем сообщений. Данный модуль разработан в компании~\company и
приведен на диаграмме для полноты картины.

Типовые сценарии работы с программой представлены на диаграмме последовательности. Слева изображена возможная
последовательность вызовов при тестировании устройств.

Справа изображен типовой сценарий работы с программой просмотра журнала.

Более подробно процесс тестирования устройств представлен на схемах программы.
Слева представлена схема тестирования БИНС, справа -- алгоритм тестирования подключенных устройств.

Окно программы тестирования устройств представлено на заключительном плакате. При запуске программы происходит настройка
элементов графического интерфейса и добавление подключенных устройств в дерево устройств. После пользователь выбирает
устройства для тестирования. После нажатия кнопки <<Старт>> осуществляется проход по дереву устройств. Для тестирования
каждого устройства используется соответствующий модуль. После завершения процесса тестирования результаты записываются в
журнал.

Схема тестирования БИНС представлена слева. Практически все компоненты системы, за исключением принтера и локальной сети
тестируются аналогично БИНС.

В начале тестирования устройства производится проверка корректности настроек и параметров подключения устройств. Далее
осуществляется чтение сообщений от устройства. После получения сообщений статуса и сообщения с данными, собранными
устройством, производится анализ полей этих сообщений.

В процессе анализа, полученная информация сохраняется для дальнейшей записи в журнал, которую осуществляет объект класса
\texttt{OffLineFuncControl}.

На заключительном плакате представлены скриншоты программы тестирования и программы для работы с журналом.

В процессе разработке были реализованы следующие возможности.

В дальнейшем планируется реализовать.

Данная система может быть впоследствии использована в составе комплекса автоматизации АРМ КМУ.
