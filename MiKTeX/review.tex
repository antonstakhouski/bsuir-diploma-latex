% Содержимое данного документа позаимсвовано из Приложения Ж из документа http://www.bsuir.by/m/12_113415_1_66883.pdf

\thispagestyle{empty}

\begin{singlespace}

{\small
  \begin{center}
    \begin{minipage}{1.0\textwidth}
      \begin{center}
        {\normalsize РЕЦЕНЗИЯ}\\[0.2cm]
        на дипломный проект студента факультета компьютерных систем и сетей Учреждения образования <<Белорусский государственный университет информатики и радиоэлектроники>>\\
        Стаховского Антона Владимировича на тему: \\
	      <<Система функционального контроля технических средств комплекса машин управления артиллерийского
	      дивизиона>>
      \end{center}
    \end{minipage}\\
  \end{center}

  Студент Стаховский~А.~\,В. выполнил дипломный проект на шести листах графического материала и на
  ~\pageref*{LastPage} страницах расчетно-пояснительной записки.

  Тема проекта является актуальной и посвящена разработке системы, позволяющей автоматизировать процесс тестирования
	подключенных устройств и облегчить ведение отчетности.
  Разработка данной системы обусловлена необходимостью повышения качества работы операторов автоматизированных рабочих
	мест.

  Разработанная система позволяет тестировать широкий перечень устройств. Система
	осуществляет журналирование результатов тестирования и предоставляет интерфейс для работы с журналом.
	Разработанная система полностью соответствует поставленному заданию.

  Пояснительная записка построена логично и последовательно отражает все этапы разработки в соответствии с календарным планом.

  В проекте приведен глубокий аналитический обзор научно-технической литературы, где рассмотрены все
	вопросы, касающиеся темы проекта.

  Разработанные алгоритмы тестирования устройств позволяют получить всю информацию о статусе устройств и их подсистем,
	извлечь данные приборов, проверить достоверность полученных данных.

  По каждому разделу и в целом по дипломному проекту приведены аргументированные выводы.

  Считаю, что представленные материалы могут быть использованы при разработке систем автоматизации комплекса машин
	управления артиллерийского дивизиона.

  Замечания:
  \begin{itemize}
	  \item в руководстве пользователя пользователя (с. 68) описана установка программы путем запуска бинарного
		  файла, в то время как на диске содержится архив с исходными файлами;
	  \item блок журналирования имеет смысл разбить на два блока, так как модули журналирования и модули просмотра
		  журнала не взаимодействуют друг с другом.
  \end{itemize}

  Пояснительная записка и графический материал оформлены аккуратно и в соответствии с требованиями ЕСКД. Информация в
  пояснительной записке изложена последовательно и структурировано.

  В целом дипломный проект выполнен технически грамотно, в полном соответствии с техническим заданием на проектирование
	и заслуживает оценки десять баллов, а дипломник Стаховский~А.\,В. --- присвоения квалификации
	<<инженер-системотехник>>.

  \vfill
  \noindent
  \begin{minipage}{0.4\textwidth}
      \setlength{\parindent}{7ex}Рецензент:\\
      канд. техн. наук, профессор\\
      УО <<Академия МВД РБ>> \\
      12.06.2018
  \end{minipage}
  \begin{minipage}{0.58\textwidth}
    \begin{flushright}
	    \textit{Подпись}~~~ Л.\,А.~Поплавская \\
    \end{flushright}
  \end{minipage}
}

\end{singlespace}
\clearpage
