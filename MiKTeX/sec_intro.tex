\sectioncentered*{Введение}
\addcontentsline{toc}{section}{Введение}
\label{sec:intro}

Защита своих границ и граждан -- одна из наиболее приоритетных задач любого государства.
Страна, которая не уделяет достаточного внимания состоянию своих войск и вооружения не может гарантировать безопасность своих граждан и сохранениее дальнейшее сохранение суверенитета.

За последние десятилетия военная техника и вооружение ушли далеко вперед.
Стали широко применяться различные датчики, спутниковые системы навигации, компьютерные сети, портативные компьютеры.
Благодаря автоматизации расчетов, настройки оборуования, тестирования переферийных устройств эффективность вооруженных сил значительно возрасла.
Во время проведения боевых действий внезапный отказ технических средств и локальной вычислительной сети может привести к серьезным потерям личного состава и потере преимущества на местности.
В таких условиях автоматизация процессов проведения тестирования является одной из наиболее приоритетных задач.

Исключительную важность во время проведения боевых действий представляет комплекс машин управления огнем, который служит для управления офицерским составом деятельностью своих подчиненных.
В состав комплекса машин управления огнем входят:
\begin{itemize}
    \item машина управления командира дивизиона
    \item командно-штабная машина дивизиона
    \item машина управления командира батареи
    \item машина управления старшего офицера батареи
    \item комплект средств управления самоходных артиллерийских орудий
\end{itemize}

Целью данного дипломного проекта является разработка и реализация системы автоматизации процессов тестирования технических средств и каналов обмена данными в локальной сети.
Программный модуль обеспечивает тестирование средств автоматизации, локальной вычислительной сети, тестирование и настройку средств связи, тестирование и настройку средств измерений, ведение и просмотр неисправностей технических средств, возникающих в процессе работы, ведение и просмотр журнала тестирования каналов связи.
Данная система в первую очередь ориентированна на использование артиллерийским дивизионом, но при небольших доработках программные модули могут быть также использованы в решениях для других армейских подразделений.

Для успешного выполнения поставленной цели, работа над проектом была разбита на следующие задачи:
\begin{itemize}
    \item выбор технологий, удовлетворяющих требованиям заказчика
    \item разработка системы функционального контроля средств связи
    \item разработка системы настройки и тестирования метеокомплекта
    \item разработка системы функционального тестирования навигационной системы
    \item разработка системы фукнкионального контроля локальной сети
    \item разработка системы системы настройки и тестирования системы синхронизации времени
\end{itemize}

Система состоит из нескольких модулей, каждый из которых тестирует определенный блок системы.
Подключение к внешним устройствам осуществляется через интерфейс RS-232.
Взаимодействие между машинами может происходить как через локальную сеть, так и через радиоканал.
