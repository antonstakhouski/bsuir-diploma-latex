\section{Функциональное проектирование}
\label{sec:func}

\subsection{Класс \texttt{QTestPrinter}}
Класс \texttt{QTestPrinter} является реализацией блока тестирования специального принтера.
Данный класс выполняет поиск подключенных принтеров, проверку правильности настроек принтера,
позволяет осуществить проверку состояния принтера путем печати тестовой страницы. Пример тестовой страницы приведен на
рис.\ref{fig:func:sample_page}.
\begin{figure}[!htb]
	\centering
	\includegraphics[scale=1.0]{sample_page}
	\caption{Пример тестовой страницы}
	\label{fig:func:sample_page}
\end{figure}

Класс \texttt{QTestPrinter} включает в себя следующие методы:

\begin{enum}
	\item Метод \texttt{test} проводит процедуру тестирования принтера. Данный метод принимает ссылку на строку
		\texttt{sRez}. Данная строка служит для логгирования процесса тестирования устройства. Метод \texttt{test} производит
		проверку наличия принтера и корректности настроек его подключения. В случае, если принтер подключен корректно, создается
		диалоговое окно для печати тестовой страницы. Для формирования тестовой страницы служит метод \texttt{print}, который
		связан с сигналом \texttt{paintRequested}. Сигнал \texttt{paintRequested} посылается при необходимости стандартного
		класса Qt \texttt{QPrintPreviewDialog} сгенерировать изображения для предпросмотра печатаемых страниц.

	\item Метод \texttt{print} осуществляет формирование тестовой страницы и настройку параметров печати. \texttt{print}
		принимает в качестве параметра указатель на класс \texttt{QPrintPreviewDialog}, который используется для настройки
		параметров печати.
\end{enum}

\subsection{Класс \texttt{VTest\_BINS3}}
Данный класс отвечает за тестирование БИНС, работающей по протоколу БИНС-3.

Класс \texttt{VTest\_BINS3} имеет следующие методы и внутренние переменные:
\begin{enum}
	\item Переменная \texttt{m\_bReceiveMess81} является флагом, который выставляется при получении сообщения
		статуса БИНС-3. Сообщение статуса имеет идентификатор 0x81.

	\item Переменная \texttt{m\_bReceiveMess01} - флаг, отвечающий за получения сообщения с навигационными данными.
		Данное Сообщение имеет идентификатор 0x01.

	\item Метод \texttt{onReadFromBins} принимает полученную с БИНС дейтаграмму в виде \texttt{QByteArray}.
		и осуществляет определение типа полученной команды. После определения типа команды устанавливается
		соответствующий флаг.

	\item Метод \texttt{onReadFromSocket} выполняет чтение данных из сокета, через который идет общение с БИНС,
		метод вызывается при появлении информации для чтения с БИНС(сигнал \texttt{readyRead}).
		Данный метод принимает полученную с БИНС дейтаграмму в виде \texttt{QByteArray}.
		При считывании байт команды дейтаграмма передается в метод \texttt{onReadFromBins} для определения типа
		команды.

	\item Метод \texttt{test} является основным методом класса \texttt{VTest\_BINS3}. Данный метод отвечает
		непосредственно за управление тестированием БИНС. Метод проверяет подключение БИНС, настройки и
		параметры подключения устройства. При успешном получении сообщения с навигационными данными и сообщения
		статуса, программа передает управление функциям \texttt{decodeMes01} и \texttt{decodeMes81} для
		получения более подробных данных о состоянии устройства.

	\item Метод \texttt{decodeMes01} отвечает за анализ сообщений с навигационными данными. Метод принимает
		дейтаграмму \texttt{sRez} в качестве входного параметра.
		Следующая информация может быть получена, используя метод
		\texttt{decodeMes01}:
		\begin{itemize}
			\item текущий курсовой угол;
			\item текущий путевой угол;
			\item текущий угол крена;
			\item текущий угол тангажа;
			\item модуль текущей скорости машины;
			\item длина пройденного пути;
			\item текущее время;
			\item используемая система координат;
			\item используемая система отсчета высоты;
			\item используемые единицы для измерения высоты;
			\item метод получения текущих координат;
			\item источник начальных координат;
			\item достоверность СНС;
			\item достоверность даты и времени;
			\item достоверность текущих углов ориентации;
			\item достоверность текущих координат.
		\end{itemize}

	\item Метод \texttt{decodeMes81} позволяет получить подробную информацию о состоянии системы. Метод принимает
		ссылку на флаг исправности устройства \texttt{bIspr} и ссылку на дейтаграмму \texttt{sRez}.
		Метод позволяет получить следующую информацию и данные:
		\begin{itemize}
			\item признак калибровки ДПЦ-2;
			\item признак исправности ДПЦ-2;
			\item состояние питания ДПЦ-2;
			\item состояние ДПЦ-2;
			\item используемые спутники;
			\item наличие навигационного решения;
			\item признак исправности БИС-3;
			\item состояние БИС-3;
			\item состояние питания БИС-3;
			\item статус обновления начального положения;
			\item корректность предыдущего выключения питания INS;
			\item фиксация срыва выставки;
			\item выполнение гирокомпассирования после выставки;
			\item статус готовности INS к решению навигационной задачи;
			\item статус проведения юстировки INS;
			\item режим навигации;
			\item состояние режима 1;
			\item признак исправности INS;
			\item режим работы устройства;
			\item время до завершения операции;
			\item наличие неисправностей в работе ДПЦ-2;
			\item наличие неисправностей в работе БИС-3;
			\item наличие неисправностей в работе блока питания;
			\item наличие неисправностей в работке канала X гироскопа;
			\item наличие неисправностей в работке канала Y гироскопа;
			\item наличие неисправностей в работке канала Z гироскопа;
			\item наличие неисправностей в работке блока акселерометров.
		\end{itemize}

	\item Метод \texttt{getCurrentTime} служит для получения текущего времени.

	\item Сигнал \texttt{getAllMessFromBINS} извещает о том, что уже получено и сообщение статуса, и сообщение с
		навигационными данными.. Данный сигнал посылается из функции \texttt{onReadFromBins} при наличии
		одновременно установленных флагов
		\texttt{m\_bReceiveMess01} и \texttt{m\_ReceiveMess81}. Данный сигнал используется в методе
		\texttt{test} при асинхронном ожидании окончания чтения из БИНС.

	\item Переменные \texttt{m\_mes01} и \texttt{m\_mes81} являются переменными типов \texttt{\_KDG\_01} и
		\texttt{\_KDG\_81} соответственно. Данные типа представляют собой структуры, описанные в заголовочном
		файле \texttt{Protocol\_BINS3.h}. Переменные данных типов инкапсулируют в себе дейтаграммы
		соответствующего формата. Переменные \texttt{m\_mes01} и \texttt{m\_mes81} используются в функциях
		\texttt{decodeMes01} и \texttt{decodeMes81} соответственно для удобного доступа к различным полям в
		заголовках дейтаграмм.
\end{enum}

\subsection{Блок журналирования}
Данный блок осуществляет ведение журнала функционального тестирования устройств. Компоненты блока служат для
формирования журнала в формате JSON.
Ниже представлен пример страницы журнала, в которой описаны результаты тестирования, записываемые в ходе одного месяца.
\medskip
\begin{minted}{JSON}
[
	{
		"date": "2018-05-04 09:16:52",
		"devices":[
			{ "name":"АРМ-К (сервер)", "result":"Неисправно", "info":"Обмен пакетами с 192.168.1.2 по 32 байт:Превышен интервал ожидания для запроса. \n \nСтатистика Ping для 192.168.1.2: \n Пакетов: отправлено = 1, получено = 0, потеряно = 1 (100% потерь),", "hasError":true },

			{ "name":"АРМ-О", "result":"Неисправно", "info":"Обмен пакетами с 192.168.1.3 по 32 байт:Превышен интервал ожидания для запроса. \n \nСтатистика Ping для 192.168.1.3: \n Пакетов: отправлено = 1, получено = 0, потеряно = 1 (100% потерь),", "hasError":true },

			{ "name":"АРМ-ФКУ", "result":"Неисправно", "info":"Обмен пакетами с 192.168.1.4 по 32 байт:Превышен интервал ожидания для запроса. \n \nСтатистика Ping для 192.168.1.4: \n Пакетов: отправлено = 1, получено = 0, потеряно = 1 (100% потерь),", "hasError":true },

			{ "name":"Принтер", "result":"Исправно", "info":"", "hasError":false }
		]
	}
	,
	{
		"date": "2018-05-04 09:33:12",
		"devices":[
			{ "name":"Радиостанция Р-181-50/50ВУ-2", "result":"Неисправно", "info":"Радиостанция Р-181-50/50ВУ-2 Устройство не подключено\n", "hasError":true }
		]
	}
]

\end{minted}

\subsubsection{Структура \texttt{DeviceInfo}}
Структура \texttt{DeviceInfo} представляет собой запись о результатах тестирования одного устройства.
Реализация структуры \texttt{DeviceInfo} представлена ниже.
\medskip
\begin{minted}{c++}
struct DeviceInfo {
        DeviceInfo();
        QString deviceName;
        QString resultMessage;
        QString additionalInfo;
        bool hasError;
};
\end{minted}
\medskip


Данная структура содержит следующие поля:
\begin{enum}
	\item \texttt{deviceName} -- поле, в котором хранится имя тестируемого устройства;
	\item \texttt{resultMessage} -- поле, в котором хранится информация о результатах тестирования;
	\item \texttt{additionalInfo} -- поле, предназначенное для хранения дополнительной информации о
		результатах тестирования устройства;
	\item \texttt{hasError} -- флаг, который указывает на наличие ошибок, обнаруженных в ходе
		тестирования устройства.
\end{enum}

\subsubsection{Класс \texttt{JournalEntry}}
Данный класс используется для формирования записи о результатах тестирования устройств.

Данный класс содержит следующие методы и переменные класса, в нем также объявлены следующие структуры:
\begin{enum}
	\item Переменная \texttt{date} служит для хранения информации о времени проведения тестирования.

	\item Переменная \texttt{devices} является списком, хранящим структуры \texttt{DeviceInfo}. Таким образом,
		данный список хранит информацию о результатах последнего тестирования устройств.

	\item Метод \texttt{addDevice} осуществляет добавление нового устройства к списку \texttt{devices}. Данный метод
		принимает в качестве параметров структуру типа \texttt{DeviceInfo}.

	\item Метод \texttt{getDate} служит для получения значения переменной \texttt{date}.

	\item Метод \texttt{getDevices} возвращает список устройств \texttt{devices}.
\end{enum}

\subsubsection{Класс \texttt{Journal}}
Класс \texttt{Journal} служит для записи результатов тестирования в журнал. Для удобства операций с журналом
тестирования, результаты тестирования хранятся в формате JSON. Класс \texttt{Journal} содержит следующие методы:
\begin{enum}
	\item Метод \texttt{store} служит для добавления новой записи в журнал тестирования. Перед добавлением новой
		записи \texttt{JournalEntry} в файл, она конвертируется в JSON строку c помощью метода \texttt{asJSON}.
		Перед записью в файл, происходит считывание всего файла, после чего новая запись добавляется к массиву
		записей с помощью метода \texttt{appendEntryToArray}.

	\item Метод \texttt{asJSON} принимает запись типа \texttt{JournalEntry} и преобразует ее в JSON строку.
		Все данные структуры, включая имена полей, преобразуются в удобный для чтения текст. Также в
		результирующую строку добавляется информация о дате проведения тестирования.

	\item Метод \texttt{appendEntryToArray} принимает ссылки на журнал тестирования \texttt{jsonArray} и новую
		запись \texttt{jsonJournalEntry}. Данный метод осуществляет добавление новой записи в массив
		\texttt{jsonArray}.
\end{enum}

\subsection{Класс \texttt{OffLineFuncControl}}
Данный класс представляет собой управляющий модуль. \texttt{OffLineFuncControl}осуществляет взаимодействие с элементами графического
интерфейса, а также с классами, отвечающими за тестирование периферийных устройств.
Данный класс включает в себя следующие методы и переменные:
\begin{enum}
	\item Переменная \texttt{m\_pTreeDevice} является указателем на дерево подключенных устройств.
	\item Переменная \texttt{m\_pBtStart} является кнопкой, запускающей процесс тестирования устройств.
	\item Переменная \texttt{m\_pBtPrint} -- кнопка для вывода результатов тестирования на экран.
	\item Переменная \texttt{m\_pBtSettings} -- кнопка для вызова настроек.
	\item Переменная \texttt{m\_pBtJournal} -- кнопка для открытия журнала тестирования.
	\item Переменная \texttt{m\_pBtTestKS} -- кнопка для начала процесса тестирования ЛВС.
	\item Переменная \texttt{m\_pBtExit} -- кнопка выхода из программы.

	\item Переменная \texttt{m\_Menu} служит для настройки элементов меню.
	\item Переменная \texttt{m\_pActInfo} является триггером для вызова окна с информацией о подробных результатах
		тестирования устройства.

	\item Переменная \texttt{m\_sockReceiveUpdate} -- сокет для получения сообщений.

	\item Переменная \texttt{m\_pActSetParam} -- триггер для управления фоном в процессе тестирования.
	\item Переменная \texttt{m\_pMenuSNS} -- триггер для добавления меню выбора режима работы СНС.
	\item Переменная \texttt{m\_pActSNS\_GPS\_GLONASS} -- триггер для добавления режима GPS/GLONASS в меню выбора
		СНС.
	\item Переменная \texttt{m\_pActSNS\_GPS} -- триггер для добавления режима GPS в меню выбора СНС.
	\item Переменная \texttt{m\_pActSNS\_GLONASS} -- триггер для добавления режима GLONASS в меню выбора СНС.

	\item Метод \texttt{messageReceived} служит для анализа полученных команд и выполнения соответствующих действий
		в ответ на них.

	\item Метод \texttt{resizeEvent} вызывается при изменении размеров окна. Данный метод служит для оптимального
		изменения положения элементов окна в соответствии с текущим размером окна.

	\item Метод \texttt{closeEvent} служит для переопределения стандартного метода закрытия окна. Данный метод
		служит для отправки служебных сообщений имитаторам устройств перед завершением программы.

	\item Метод \texttt{loadDevices} используется для добавления устройств в список подключенных устройств.

	\item Метод \texttt{receiveSignalUpdate} срабатывает после получения сигнала \texttt{readyRead} от
		\texttt{m\_sockReceiveUpdate}. Данный метод служит для получения дейтаграмм. После получения каждой
		дейтаграммы управление передается методу \texttt{loadDevices} для дальнейшей обработки.

	\item Метод \texttt{onTestARM} проверяет корректность номера текущего АРМ и передает управление методу
		\texttt{test} класса \texttt{QTestARM} для проведения тестирования АРМ.

	\item Метод \texttt{onChangeCheckDevice} используется при изменении состояния чекбокса, отвечающего за включение
		устройства в список для тестирования.

	\item Метод \texttt{onStart} вызывается при нажатии кнопки \texttt{m\_pBtStart}. Данный метод служит для начала
		процесса функционального тестирования выбранных устройств.

	\item Метод \texttt{onMenu} срабатывает при вызове меню. Данный метод служит для вывода определенных пунктов
		меню, которые зависят от типа конкретного устройства.

	\item Метод \texttt{onInfo} срабатывает при выводе информационного сообщения. Данный метод вызывается при
		срабатывании триггера \texttt{m\_pActInfo}. Метод получает id текущего устройства и передает выполнение
		методу \texttt{onInfoDevice}.

	\item Метод \texttt{onInfoDevice} связан с кнопкой \texttt{pBtInfo}. Данный метод служит для предоставления
		информации о выбранном устройстве.

	\item Метод \texttt{onPrint} вызывает диалоговое окно для печати тестовой страницы.

	\item Метод \texttt{onJournal} служит для вызова внешней программы, с помощью которой осуществляется просмотр
		содержимого журнала тестирования.

	\item Метод \texttt{print} служит для вывода результатов тестирования устройств на печать.

	\item Метод \texttt{onSettings} вызывает программу для настройки параметров АРМ.

	\item Метод \texttt{onTestKS} вызывает программу для тестирования компьютерной сети.

	\item Метод \texttt{saveRezTest} служит для сохранения результатов тестирования устройств.
\end{enum}

\subsection{Класс \texttt{VTest\_R181}}
Данный класс используется для тестирования работы радиостанции. Модуль проверяет подключение устройства, получает
информацию о состоянии радиостанции.

Класс \texttt{VTest\_R181} включает в себя следующие методы и переменные:
\begin{enum}
	\item Метод \texttt{getParamsRS} позволяет получить параметры радиостанции.

	\item Метод \texttt{test} проводит тестирование подключенной радиостанции. Метод проверяет настройки подключения
		радиостанции, а также позволяет получить текущие следующую информацию о состоянии компонентов
		радиостанции:
		\begin{itemize}
			\item тип радиостанции;
			\item собственный адрес станции;
			\item номер канала;
			\item частота передачи;
			\item частота приема;
			\item выходная мощность;
			\item состояние шумоподавителя.
		\end{itemize}

	\item Переменная \texttt{m\_protR181\_LS} служит для управления соединением с радиостанцией, обмена сообщениями
		с ней.

	\item Переменная \texttt{m\_rsSett} используется для хранения настроек и параметров радиостанции.

	\item Переменная \texttt{m\_ownAddr} служит для хранения адреса радиостанции.
\end{enum}

\subsection{Класс \texttt{VTest\_MeteoWXT520}}
Класс \texttt{VTest\_MeteoWXT520} используется для тестирования работы метеостанции. В ходе тестирования проверяются не
только корректность подключения и параметры метеостанции, также происходит и запись текущих метеоданных в журнал.

Данный класс включает в себя следующие переменные и методы:
\begin{enum}
	\item Метод \texttt{test} проводит проверку настроек и параметров подключения метеостанции. Проводится установка
		параметров для общения с метеостанцией, используя последовательный порт. Данный метод позволяет получить
		следующую информацию и извлечь следующие данные:
		\begin{itemize}
			\item параметры подключения метеостанции;
			\item адрес устройства;
			\item минимальная скорость ветра;
			\item средняя скорость ветра;
			\item максимальная скорость ветра;
			\item минимальное направление ветра;
			\item среднее направление ветра;
			\item максимальное направление ветра;
			\item атмосферное давление;
			\item температура воздуха;
			\item относительная влажность воздуха;
			\item накопление дождя;
			\item продолжительность дождя;
			\item интенсивность дождя;
			\item пиковая интенсивность дождя;
			\item накопление града;
			\item продолжительность града;
			\item интенсивность града;
			\item пиковая интенсивность града;
			\item температура подогрева;
			\item напряжение подогрева;
			\item напряжение питания;
			\item опорное напряжение 3,5 В.
			\item данные о интенсивности и продолжительности дождя;
			\item данные о интенсивности и продолжительности града;
			\item данные о напряжении в различных компонентах метеостанции.
		\end{itemize}
	\item Метод \texttt{getRezLastTest} служит для получения результатов последнего тестирования устройства.
	\item Переменная \texttt{m\_lastInfo} служит для хранения результатов последнего тестирования.
\end{enum}

\subsection{Класс \texttt{VTest\_GPS}}
Класс \texttt{VTest\_GPS} используется для проведения тестирования подключенного навигационного приемника.

Класс \texttt{VTest\_GPS} включает в себя следующие переменные и методы:
\begin{enum}
	\item Метод \texttt{test} производит проверку настроек и параметров подключения устройства, также данный метод
		позволяет получить данные о текущем местоположении и получить информацию о количестве используемых
		спутников. Данный метод также позволяет проверить корректность полученных данных.
\end{enum}


\subsection{Класс \texttt{VTest\_Kaponir}}
Класс \texttt{VTest\_Kaponir} содержит в себе методы, позволяющие произвести функциональное тестирование прибора
наблюдения разведчика <<Капонир>>.

Данный класс включает в себя следующие переменные и методы:
\begin{enum}
	\item Метод \texttt{test} служит для проверки параметров и настроек подключения устройства и получения
		информации о состоянии устройства. Данный метод позволяет получить следующую информацию и извлечь
		следующие данные:
		\begin{itemize}
			\item параметры подключения устройства;
			\item заводской номер прибора;
			\item режим работы прибора;
			\item состояние работы дальномера;
			\item статус работы компаса;
			\item состояние GPS;
			\item уровень заряда батареи;
			\item режим работы навигационной системы;
			\item достоверность полученных координат;
			\item тип используемой СНС;
			\item используемая система координат;
			\item количество используемых спутников;
			\item геодезические координаты;
			\item корректность подключения прибора.
		\end{itemize}

	\item Метод \texttt{queryServiceKaponir\_Multicast} служит для запроса данных от прибора, используя для этого
		мультикаст рассылку.

	\item Метод \texttt{onReceiveMulticast} служит для чтения дейтаграмм, получаемых через мультикаст рассылку.

	\item Переменная \texttt{m\_sockSend} служит создания сокета, выполняющего отправку сообщений прибору
		<<Капонир>>.

	\item Переменная \texttt{m\_sockReceive} служит для создания сокета, выполняющего прием сообщений от прибора
		<<Капонир>>.

	\item Переменная \texttt{m\_BindSock} используется для проверки статуса операции bind.

	\item Переменная \texttt{m\_baReceive} служит для хранения полученных дейтаграмм.
\end{enum}

