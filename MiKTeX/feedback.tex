% Содержимое данного документа позаимсвовано из Приложения Е из документа http://www.bsuir.by/m/12_113415_1_66883.pdf

\thispagestyle{empty}

\begin{singlespace}

{\small
  \begin{center}
    \begin{minipage}{0.8\textwidth}
      \begin{center}
        {\normalsize ОТЗЫВ}\\[1em]
        на дипломный проект студентки факультета информационных технологий
        и управления Учреждения образования <<Белорусский государственный университет информатики и радиоэлектроники>>\\
        Москаленко Ольги Николаевны \\
        на тему: <<Система передачи данных>>
      \end{center}
    \end{minipage}
  \end{center}

На время дипломного проектирования перед студенткой Москаленко~О.\,Н. была поставлена задача разработать высокоскоростную систему передачи данных по занятым телефонным линиям.
Тема является актуальной, т.\,к. многие абоненты, имеющие дома компьютеры, для выхода на коллективные сети передачи данных имеют только телефонную линию связи, по которой могут вестись интенсивные разговоры.
Проблема <<последней мили>> при разработке высоконадежных систем передачи данных является основной при создании подобных систем.

Москаленко~О.\,Н. на основании анализа большого количества специализированной литературы произвела выбор частотного диапазона для передачи данных в обоих направлениях и предложила для повышения достоверности передачи информации применить решающую обратную связь.

В процессе проектирования были разработаны алгоритмы функционирования, структурные и принципиальные схемы.
Система разработана на современной элементной базе с использованием pic контроллеров.

Приведенные расчеты и программное обеспечение "--- это результат высокоэффективной работы над темой и умения использовать техническую литературу и применять на практике знания, полученные за годы обучения в университете.

Работа над проектом велась ритмично и в соответствии с календарным графиком.
Пояснительная записка и графический материал оформлены аккуратно и в соответствии с требованиями ЕСКД.

Результаты, полученные в дипломном проекте, использованы в разработке системы передачи дискретной информации, которая рекомендована к серийному выпуску, о чем свидетельствует Акт внедрения, прилагаемый к пояснительной записке.

Дипломный проект Москаленко~О.\,Н. соответствует техническому заданию и отличается глубокой проработкой темы и выполнен с применением современных прогрессивных технологий.

Считаю, что Москаленко~О.\,Н. освоила технику инженерного проектирования технических систем, подготовлена к самостоятельной работе по специальности 1-53~01~07
<<Информационные технологии и управление в технических системах>> и заслуживает присвоения квалификации инженера по информационным технологиям и управлению.

  \vfill
  \noindent
  \begin{minipage}{0.54\textwidth}
    \begin{flushleft}
      Руководитель проекта:\\
      д-р техн. наук, начальник сектора \\
      информационных технологий НАН Беларуси\\
      23.01.09
    \end{flushleft}
  \end{minipage}
  \begin{minipage}{0.44\textwidth}
    \begin{flushright}
      \underline{\hspace*{3cm}} М.\,Н.~Реут
    \end{flushright}
  \end{minipage}
}

\end{singlespace}

\clearpage
