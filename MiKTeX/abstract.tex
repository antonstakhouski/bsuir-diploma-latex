\sectioncentered*{Реферат}
\thispagestyle{empty}

Дипломный проект представлен следующим образом. Электронные
носители: 1 компакт-диск. Чертежный материал: 6 листов формата А1.
Пояснительная записка: \pageref*{LastPage} страницы, \totfig{}~рисунков, \tottab{}~таблицы, \totref{}~
литературных источников, 4 приложения.

Ключевые слова: функциональный котроль, техническое средство,\break АРМ, БИНС, КМУ, метеостанция, радиостанция, принтер, артиллерийский дивизион.

Целью дипломного проекта является разработка удобного в использовании инструмента, для тестирования и настройки
технических средств и локальной вычислительной сети комплекса машин управления артиллерийского дивизиона.

При разработке дипломного проекта были использованы: библиотека Qt, библиотеки протоколов и программы имитаторы, разраработанные в
компании~\company.

Разработанный проект ориентирован на использование в составе комплекса автоматизации комплекса машин управления
артиллерийского дивизиона.

В разделе технико-экономического обоснования был произведён расчёт затрат на создание ПО, а также прибыли от разработки,
получаемой компанией.
Проведённые расчёты показали экономическую целесообразность проекта.

Дипломный проект является завершённым. Задача, поставленная в
начале разработки, решена в полном объёме. Присутствует возможность
дальнейшего расширения и развития модуля, а также увеличение
предоставляемого функционала.

\clearpage
