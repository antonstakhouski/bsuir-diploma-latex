\newcommand{\byr}{руб}

\section{Технико-экономическое обоснование разработки системы функционального контроля технических средств КМУ
артиллерийского дивизиона}

% Begin Calculations
\FPeval{\bonusRate}{1.75}

\FPeval{\leadDevHours}{55}
\FPeval{\devHours}{180}
\FPeval{\qaHours}{120}

\FPeval{\leadDevPerMonth}{2995.5}
\FPeval{\devPerMonth}{1180}
\FPeval{\qaPerMonth}{998.5}

\FPeval{\normativeManHours}{168}

\FPeval{\leadDevPerHourExact}{\leadDevPerMonth / \normativeManHours}
\FPeval{\devPerHourExact}{\devPerMonth / \normativeManHours}
\FPeval{\qaPerHourExact}{\qaPerMonth / \normativeManHours}

\FPround{\leadDevPerHour}{\leadDevPerHourExact}{3}
\FPround{\devPerHour}{\devPerHourExact}{3}
\FPround{\qaPerHour}{\qaPerHourExact}{3}

\FPeval{\leadDevSalaryExact}{\leadDevHours * \leadDevPerHourExact}
\FPeval{\devSalaryExact}{\devHours * \devPerHourExact}
\FPeval{\qaSalaryExact}{\qaHours * \qaPerHourExact}

\FPround{\leadDevSalary}{\leadDevSalaryExact}{1}
\FPround{\devSalary}{\devSalaryExact}{1}
\FPround{\qaSalary}{\qaSalaryExact}{1}

\FPeval{\totalSalaryExact}{clip(\leadDevSalaryExact + \devSalaryExact + \qaSalaryExact) * \bonusRate}

\FPround{\totalSalary}{\totalSalaryExact}{1}

\FPeval{\normExtraSalary}{15}

\FPeval{\extraSalaryExact}{\totalSalaryExact * \normExtraSalary / 100}
\FPround{\extraSalary}{\extraSalaryExact}{1}

\FPeval{\socialFeesCoeff}{34.6}
\FPeval{\socialFeesExact}{clip(\totalSalaryExact + \extraSalaryExact) * \socialFeesCoeff / 100}
\FPround{\socialFees}{\socialFeesExact}{1}

\FPeval{\extraFeesCoeff}{125}
\FPeval{\extraFeesExact}{\totalSalaryExact * \extraFeesCoeff / 100}
\FPround{\extraFees}{\extraFeesExact}{1}

\FPeval{\totalExpensesExact}{\totalSalary + \extraSalary + \socialFees + \extraFees}
\FPround{\totalExpenses}{\totalExpensesExact}{1}

\FPeval{\profitTax}{18}
\FPeval{\ndsNormative}{20}

\FPeval{\productPrice}{21000}

\FPeval{\ndsExact}{clip(\productPrice * \ndsNormative)  / clip(100 + \ndsNormative)}
\FPround{\nds}{\ndsExact}{1}

\FPeval{\cleanProfitDevExact}{\productPrice - \ndsExact - \totalExpensesExact}
\FPround{\cleanProfitDev}{\cleanProfitDevExact}{1}

\FPeval{\cleanProfitExact}{\cleanProfitDevExact - \cleanProfitDevExact * \profitTax / 100}
\FPround{\cleanProfit}{\cleanProfitExact}{1}

\FPeval{\profitLevelExact}{\cleanProfitExact / \totalExpenses * 100}
\FPround{\profitLevel}{\profitLevelExact}{1}

\FPeval{\credit}{8.5}

% End Calculations


\subsection{Описание функций, назначения и потенциальных пользователей ПО}

Система функционального контроля технических средств КМУ артиллерийского дивизиона предназначена для использования в
армейских подразделениях. Данная система используется как часть комплекса автоматизации АРМ артиллерийского дивизиона.

Разработанная система обладает следующими функциями:
\begin{itemize}
		\item тестирование АРМ;
		\item тестирование АВСКУ;
		\item тестирование специального принтера;
		\item тестирование КНС;
		\item тестирования МНСТО;
		\item тестирование БИНС.
\end{itemize}

Данный продукт разрабатывается под заказ для вооруженных сил.

\subsection{Расчет на разработку ПО}

Расчет затрат на разработку ПО рассчитывается упрощенно, на основе следующих статей~\cite{economics_diploma}:
\begin{itemize}
	\item затраты на основную заработную плату разработчиков;
	\item затраты на дополнительную заработную плату разработчиков;
	\item отчисления на социальные нужды;
	\item прочие отчисления (амортизационные отчисления, арендная плата за помещения и
		оборудование, расходы на управление и реализацию, расходы на электроэнергию и тому подобное).
\end{itemize}

Произведем расчет данных статей.

\subsubsection{Затраты на основную заработную плату разработчиков}

Затраты на основную заработную плату команды разработчиков определяются исходя из численности и состава команды,
размеров месячной заработной платы каждого из участника команды, а также общей трудоемкости разработки программного
обеспечения.


\begin{equation}
  \label{eq:econ:main_salary}
	\text{З}_\text{о} = \text{К}_\text{пр} \cdot \sum_{i = 1}^{n} \text{З}_\text{чi} \cdot t_{i} \text{\,,}
\end{equation}

\begin{explanation}
	где & $ \text{K}_\text{пр} $ & коэффициент премий; \\
	& $ \text{З}_\text{чi} $ & часовая заработанная плата i-го исполнителя, руб; \\
	& $ t_{i} $ & трудоемкость работ, выполняемых i-м исполнителем, ч; \\
	& $ n $ & количество исполнителей, занятых разработкой конкретного ПО.
\end{explanation}

Часовая заработная плата определяется путем деления месячной заработной платы на количество рабочих часов в месяце и
рассчитывается по формуле~\ref{eq:econ:hour_salary}.

\begin{equation}
  \label{eq:econ:hour_salary}
	\text{З}_\text{чi} = \frac{\text{З}_\text{мi}}{\text{Ф}_\text{рв}} \text{\,,}
\end{equation}

\begin{explanation}
	где & $ \text{З}_\text{мi} $ & месячная заработная плата i-го участника команды, руб; \\
	& $ \text{Ф}_\text{рв} $ & установленный фонд рабочего времени, \num{\normativeManHours} ч.
\end{explanation}

Средняя заработная плата по Республике Беларусь берется из открытого источника~\cite{salaries}.

Расчет затрат на основную заработную плату команды разработчиков приведен в таблице~\ref{table:econ:salaryCalc}.

\begin{table}[!ht]
\small
\caption{Расчет затрат на основную заработную плату команды разработчиков}
	\label{table:econ:salaryCalc}
	\centering
	\begin{tabular}{| >{\centering}m{0.14\textwidth}
		| >{\centering}m{0.17\textwidth}
		  | >{\centering}m{0.13\textwidth}
		  | >{\centering}m{0.12\textwidth}
		  | >{\centering}m{0.15\textwidth}
		  | >{\centering\arraybackslash}m{0.13\textwidth}|}

		\hline
		Участники команды & Вид выполняемой работы & Месячная заработная плата, р. & Часовая заработная плата, р. &
		Трудоемкость работ, ч. & Зарплата по тарифу, р. \\

		\hline
		Ведущий инженер-программист & Разработка архитектуры & \num{\leadDevPerMonth} & \num{\leadDevPerHour} &
		\num{\leadDevHours} & \num{\leadDevSalary} \\

		\hline
		Инженер-программист & Реализация программного продукта & \num{\devPerMonth} & \num{\devPerHour} &
		\num{\devHours} & \num{\devSalary} \\

		\hline
		Тестировщик & Тестирование программного продукта & \num{\qaPerMonth} & \num{\qaPerHour} & \num{\qaHours}
		& \num{\qaSalary} \\

		\hline
		\multicolumn{5}{|l|}{Премия} & \num{\bonusRate} \\

		\hline
		\multicolumn{5}{|l|}{Итого затраты на основную заработную плату разработчиков} & \num{\totalSalary} \\
		\hline
	\end{tabular}
\end{table}

\subsubsection{Затраты на дополнительную заработную плату разработчиков}

Данные затраты включают в себя выплаты, предусмотренные законодательством о труде:
\begin{itemize}
	\item оплата льготных часов;
	\item оплата времени выполнения государственных обязанностей;
	\item оплаты трудовых отпусков;
	\item выплаты, не связанные с основной деятельностью исполнителей.
\end{itemize}

Затраты на дополнительную заработную плату разработчиков определяются по следующей формуле

\begin{equation}
  \label{eq:econ:extra_salary}
	\text{З}_\text{д} = \frac{\text{З}_\text{о} \cdot \text{Н}_\text{д}}{\num{100}} \text{\,,}
\end{equation}

\begin{explanation}
	где & $ \text{З}_\text{о} $ & затраты на основную заработную плату, руб; \\
	& $ \text{Н}_\text{д} $ & норматив дополнительной заработной платы, \num{\normExtraSalary} \%.
\end{explanation}

Зарплаты на дополнительную заработную плату:

\begin{equation}
  \label{eq:econ:extra_salary_calc}
	\text{З}_\text{д} = \frac{\num{\totalSalary} \cdot \num{\normExtraSalary}}{\num{100}} = \num{\extraSalary} {\text{ \byr}} \text{.}
\end{equation}

\subsubsection{Отчисления на социальные нужды}

Отчисления на социальные нужды идут в фонд социальной защиты населения и на обязательное страхование. Эти отчисления определяются в соответствии с действующими законодательными актами по формуле

\begin{equation}
  \label{eq:econ:social_fees}
	\text{Р}_\text{соц} = \frac{(\text{З}_\text{о} + \text{З}_\text{д}) \cdot \text{Н}_\text{соц}} {\num{100}} \text{\,,}
\end{equation}

\begin{explanation}
	где & $ \text{Н}_\text{соц} $ & норматив отчислений на социальные нужды, \num{\socialFeesCoeff} \%.
\end{explanation}

Отчисления на социальные нужды:

\begin{equation}
  \label{eq:econ:social_fees_calc}
	\text{З}_\text{о} = \frac{(\num{\totalSalary} + \num{\extraSalary}) \cdot \num{\socialFeesCoeff}}{\num{100}} =
	\num{\socialFees} {\text{ \byr}} \text{.}
\end{equation}

\subsection{Прочие отчисления}

Прочие затраты включаются в себестоимость разработки ПО в процентах от затрат на основную заработную плату команды
разработчиков.

\begin{equation}
  \label{eq:econ:extra_fees}
	\text{З}_\text{пз} = \frac{\text{З}_\text{о} \cdot \text{Н}_\text{пз}} {\num{100}} \text{\,,}
\end{equation}

\begin{explanation}
	где & $ \text{Н}_\text{пз} $ & норматив прочих затрат, \num{\extraFeesCoeff} \%.
\end{explanation}

Прочие затраты составляют:

\begin{equation}
  \label{eq:econ:extra_fees_calc}
	\text{З}_\text{пз} = \frac{\num{\totalSalary} \cdot \num{\socialFeesCoeff}}{\num{100}} =
	\num{\extraFees} {\text{ \byr}} \text{.}
\end{equation}

\subsection{Полная сумма затрат}
Полная сумма затрат на разработку программного обеспечения находится путем суммирования всех рассчитанных статей затрат

\begin{table}[!ht]
	\caption{Затраты на разработку программного обеспечения}
	\label{table:econ:total_expenses}
	\centering
	\begin{tabular}{| >{\raggedright}m{0.82\textwidth}
		  | >{\centering\arraybackslash}m{0.13\textwidth}|}

		\hline
		\centering{Статья затрат} & Сумма, р. \\

		\hline
		Основная заработная плата команды разработчиков & \num{\totalSalary} \\

		\hline
		Дополнительная заработная плата команды разработчиков & \num{\extraSalary} \\

		\hline
		Отчисления на социальные нужды & \num{\socialFees} \\

		\hline
		Прочие затраты & \num{\extraFees} \\

		\hline
		Общая сумма затрат на разработку & \num{\totalExpenses} \\

		\hline
	\end{tabular}
\end{table}

\subsection{Оценка результата разработки и использования ПО}

При разработке или использовании программного продукта можно получить экономический эффект, который напрямую влияет
на экономические показатели компании, или неэкономический эффект, который напрямую не связан с экономическими
показателями деятельности компании.

\subsection{Экономический эффект}

При разработке программного обеспечения под заказ экономический эффект может быть рассчитан как для
организации-разработчика, так и для заказчика ПО. В нашем случае, экономический эффект получает только\break организация-разработчик.

Экономическим эффектом для организации-разработчика является чистая прибыль,
если организация является плательщиком налогов, либо прибыль до налогообложения, если организация освобождена от уплаты
налогов на прибыль, полученная от реализации программного обеспечения.
В нашем случае, компания является налогоплательщиком, следовательно, нужно рассчитать чистую прибыль.

Чистая прибыль для организаций, которые являются плательщиком налога на прибыль, рассчитывается по формуле

\begin{equation}
  \label{eq:econ:clean_profit}
	\text{П}_\text{ч} = \text{П} - \frac{\text{П} \cdot \text{Н}_\text{п}} {\num{100}} \text{\,,}
\end{equation}

\begin{explanation}
	где & $ \text{П} $ & прибыль до налогообложения, руб; \\
	& $ \text{Н}_\text{п} $ & ставка налога на прибыль, в соответствии с действующим законодательством,
	\num{\profitTax}\%.
\end{explanation}

Прибыль до налогообложения организации-разработчика рассчитывается по формуле

\begin{equation}
  \label{eq:econ:clean_profit_dev}
	\text{П} = \text{Ц} - \text{НДС} - \text{З}_\text{р} \text{\,,}
\end{equation}

\begin{explanation}
	где & $ \text{Ц} $ & цена реализации программного продукта заказчику, руб; \\
	& $ \text{НДС} $ & сумма налога на добавленную стоимость, руб; \\
	& $ \text{З}_\text{р} $ & затраты на разработку программного обеспечения, руб.
\end{explanation}

Сумма налога на добавленную стоимость рассчитывается для организаций, являющимися плательщиками НДС,
по следующей формуле

\begin{equation}
  \label{eq:econ:nds}
  \text{НДС} =
    \frac{ \text{Ц} \cdot \text{Н}_{\text{дс}} }
	 { \num{100}\text{\%} + \text{Н}_\text{дс} } \text{\,,}
\end{equation}

\begin{explanation}
	где & $ \text{Н}_{\text{дс}} $ & ставка налога на добавленную стоимость согласно действующему законодательству,
	\num{\ndsNormative}\%.
\end{explanation}

Цена программного продукта устанавливается на основе цен на программное обеспечение, выполняющие схожие функции.
Для данного продукта отсутствуют точные аналоги в открытом доступе, поэтому цена формировалась на основе нескольких
продуктов, предлагающих похожий функционал. В результате была установлена цена в размере \num{\productPrice} руб.

Сумма налога на добавленную стоимость составляет:

\begin{equation}
  \label{eq:econ:nds_calc}
  \text{НДС} =
	\frac{ \num{\productPrice} \cdot \num{\ndsNormative}}
	 { \num{100} + \num{\ndsNormative}} = \num{\nds} \text{ \byr.}
\end{equation}

Прибыль до налогообложения составляет:
\begin{equation}
  \label{eq:econ:clean_profit_dev_calc}
	\text{П} = \num{\productPrice} - \num{\nds} - \num{\totalExpenses}
	 = \num{\cleanProfit} \text{ \byr.}
\end{equation}

Экономический эффект с учетом налога на прибыль, рассчитанный по формуле~\ref{eq:econ:clean_profit}:
\begin{equation}
  \label{eq:econ:clean_profit_calc}
	\text{П}_\text{ч} = \num{\cleanProfitDev} - \frac{\num{\cleanProfitDev} \cdot \num{\profitTax}}{\num{100}}
	 = \num{\cleanProfit} \text{ \byr.}
\end{equation}

Для того, чтобы сделать вывод об эффективности разработки и реализации программного продукта по установленной цене,
необходимо рассчитать уровень рентабельности.
Рассчитанный уровень рентабельности сравнивается со средней ставкой по банковским депозитам.

Уровень рентабельности рассчитывается по формуле
\begin{equation}
  \label{eq:econ:profit_level}
	\text{У}_\text{р} =
	\frac{\text{П}_\text{ч}}
	 {\text{З}_\text{р}} \cdot \num{100}{\text{\%}} \text{.}
\end{equation}

Уровень рентабельности составляет
\begin{equation}
  \label{eq:econ:profit_level_calc}
	\text{У}_\text{р} = \frac{\num{\cleanProfit}}{\num{\totalExpenses}} \cdot \num{100}\text{\%}
	 = \num{\profitLevel} \text{ \%.}
\end{equation}

Средняя ставка по банковским депозитам на апрель месяц для юридических лиц составляет \num{\credit}\%.
Сравнивая уровень рентабельности и среднюю ставку по банковским депозитам можно сделать вывод,
что проект экономически эффективен, т.к. уровень рентабельности больше, чем средняя ставка по банковским депозитам.

\subsection{Неэкономический эффект}

Для организации-заказчика данный продукт приводит к получению\break только неэкономического эффекта.
При использовании данного программного продукта в составе комплекса автоматизации АРМ, с офицеров снимаются задачи по
обнаружению и настройке подключенных устройств вручную, упрощается процесс тестирования и мониторинга подключенных
\break
устройств.

\hfill
\clearpage


