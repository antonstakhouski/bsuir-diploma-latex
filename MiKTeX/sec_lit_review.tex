\section{Обзор литературы}
\label{sec:lit_review}

\subsection{Обзор существующих аналогов}
\label{sub:lit_review:analogues}
Разработанная в ходе дипломного проектировании система функционального контроля технических средств не имеет широко
известных прямых аналогов. Отсутствие конкурирующих систем вызвано тем следующими причинами:
\begin{itemize}
	\item технические средства, предназначенные для нужд армии имеют ограниченное распространение;
	\item протоколы обмена данными между техническими средствами и ЭВМ зачастую являются закрытыми;
	\item модуль функционального контроля зачастую является частью закрытого проприетарного ПО, используемого для
		автоматизации задач личного состава;
	\item разнообразие устройств, протоколов обмена.
\end{itemize}

Таким образом, ввиду невозможности поиска аналогов среди ПО для вооруженных сил, в данном разделе будут проанализированы
наиболее близкие аналоги из других областей.

Наиболее приближенным аналогом является система функционального применяемая на железных дорогах Российской федерации
~\cite{rus_rails}.

\begin{figure}[ht]
	\centering
	\includegraphics[scale=0.40]{rw_control}
	\caption{Модуль функционального контроля системы автоматизации РЖД~\cite{rus_rails}}
	\label{fig:lit_reiview:analogues:rw_control}
\end{figure}

Данная система функционального контроля позволяет проводить мониторинг различных участков железной дороги, оказывать
управляющие воздействия на объекты контроля, выводить информацию на экран или бумажный носитель, экспортировать данные в
другие программы для работы с результатами мониторинга.

Основные достоинства программы:
\begin{itemize}
	\item удобный интерфейс(использована аналогия с действующими устройствами ввода-вывода информации в
		железнодорожной автоматике и телемеханике));
	\item вывод информации в иерархическом виде;
	\item возможность вывода разного набора информации пользователям в зависимости от должности владельца ПК.
\end{itemize}

К недостаткам можно отнести закрытость ПО, отсутствие кроссплатформенности(поддерживается только ОС Windows).

Еще один аналог -- система функционального контроля электронной аппаратуры AX518~\cite{AX518}. Система предназначена для тестирования полупроводниковых микросхем, процессоров, ЦАП, АЦП, устройств в системе радио-идентификации, устройств поддерживающих технологии широкополосного вещания и беспроводных сетей, приемопередатчиков.
Имеется возможность для проведения измерений, как в частотной, так и во временной области, спектрального анализа, измерения RF мощностей, потерь, и шумов.

Система для автоматического тестирования электронных устройств \break AX518 состоит из 5-слотового шасси стандарта AXIe и 18-слотового шасси стандарта PXI.
Система дополняется компьютером и программным обеспечением.
В состав программного обеспечения входят стандартные библиотеки и специфические программы для тестирования определенных устройств.

Недостатком данной системы является то, что она является аппаратно-программной. Это затрудняет ее внедрение и
значительно повышает стоимость системы.

\begin{figure}[ht]
	\centering
	\includegraphics[scale=1.2]{ax518_soft}
	\caption{Программный модуль системы тестирования электронных устройств AX518~\cite{AX518}}
	\label{fig:lit_reiview:analogues:ax518_soft}
\end{figure}

\subsection{Аналитический обзор}
\label{sub:lit_review:analitics}
Технические средства --  изделия, оборудование, аппаратура и их составные части, функционирующие на основании законов электротехники, радиотехники и электроники и содержащие электронные компоненты и схемы.

КМУ -- комплекс машин управления.
В состав комплекса машин управления огнем входят:
\begin{itemize}
	\item машина управления командира дивизиона~\cite{div_car};
	\item командно-штабная машина дивизиона;
	\item машина управления командира батареи;
	\item машина управления старшего офицера батареи.
\end{itemize}

\begin{figure}[ht]
	\centering
	\includegraphics[scale=0.35]{div_com}
	\caption{Машина начальника штаба дивизиона на колесном шасси~\cite{div_car}}
	\label{fig:lit_reiview:analytics:div_com}
\end{figure}
В одном дивизионе имеется несколько машин разного уровня управления, содержащих в своем составе разные ТС.
Например, метеокомплект стоит только на нескольких машинах, радиостанции имеются в каждой машине, бесплатформенная
инерциальная навигационная система присутствует также на каждой машине, но имеют разные типы устройства,
ЛВС присутствует в каждой машине.
Программное обеспечение написано для всех машин КМУ с возможностью выборки подключенных ТС.

Разработанное в ходе дипломного проектирования программное обеспечение предназначено для развертывания в подвижном
комплексе средств автоматизации управления~\cite{patent_2263960}.
Этот подвижный комплекс средств автоматизации управления, размещенный в подвижном объекте на шасси автомобиля повышенной
грузоподъемности, содержит несколько автоматизированных рабочих места(АРМ) должностных лиц, размещенных в кузове-фургоне
подвижного объекта, оборудованных средствами вычислительной техники и средствами передачи данных, радиорелейную станцию
с антеннами, коротковолновую радиостанцию, две ультракоротковолновые радиостанции, локальную вычислительную
сеть, специальный принтер.

\begin{figure}[ht]
	\centering
	\includegraphics[scale=0.45]{arm}
	\caption{Автоматизированное рабочее место~\cite{patent_2263960}}
	\label{fig:lit_reiview:analytics:arm}
\end{figure}

Программа функционального контроля предназначена для осуществления автоматизации процессов проведения тестирования
технических\break средств.
Программа функционального контроля обеспечивает выполнение\break следующих функций:
\begin{enumerate}
\item тестирование средств автоматизации, локальной вычислительной\break сети(ЛВС), визуализацию информации о доступных в ЛВС автоматизированных рабочих местах(АРМ);
\item тестирование и настройку средств связи;
\item тестирование и настройку средств измерения.
\end{enumerate}

\subsection{Интегрированный навигационно-информационный комплекс}
\label{sub:lit_review:ins}

Интегрированный навигационно-информационный комплекс это комплексная бесплатформенная система ориентации и навигации,
построенная с использованием высокоточных акселерометров и волоконно-оптических гироскопов с замкнутым контуром,
построена на принципе комплексирования данных бесплатформенной инерциальной системы с одометром, приемником спутниковых навигационных сигналов GPS/GLONASS и приемником барометрического давления.

Аппаратура позволяет  решать весь комплекс задач топопривязки, навигации и ориентирования средств ракетных войск и артиллерии в любое время в любых погодных условиях независимо от доступности сигналов СНС:
\begin{itemize}
	\item определение текущих координат объекта на стоянке (огневой позиции, районе сосредоточения), в ходе
		совершения марша;
	\item определение углов ориентации подвижных объектов (азимутального угла продольной оси машины, углов крена и
		тангажа);
	\item автоматическое определения по сигналам спутниковых навигационных систем GPS/GLONASS текущего
		единого и местного времени с использованием поправок на часовой пояс;
	\item отображение на мониторе бортовой ЭВМ, индикаторных панелях текущих значений координат и высоты, а также
		скорости, угла продольной оси машины;
	\item отображение на мониторе бортовой ЭВМ навигационной и топогеодезической информации на электронных картах
		местности собственного местоположения.
\end{itemize}

Состав интегрированного навигационно-информационного комплекса:
\begin{itemize}
	\item блок спутниковый навигационный с цифровым датчиком атмосферного давления;
	\item цифровой одометрический датчик пути;
	\item блок управления и обработки данных;
	\item блок инерциальный навигационный измерительный на основе высокоточных акселерометров и волоконно-оптических
		гироскопов.
\end{itemize}

Ниже приведен краткий обзор компонентов комплекса INS.

\subsubsection{Бортовая ЭВМ}
\label{sub:lit_review:ins:evm}
ПЭВМ серии БК402 служит для выполнения вычислительных функций в подвижных объектах на гусеничном и колесном ходу в закрытых кузовах
как на стоянке так и в движении. Возможно также использование ПЭВМ в стационарных условиях~\cite{bk402}.
\begin{figure}[ht]
	\centering
	\includegraphics[scale=2.0]{bk402}
	\caption{Бортовая персональная электронная вычислительная машина серии БК402~\cite{bk402}}
	\label{fig:lit_reiview:ins:evm:bk402}
\end{figure}

Состав ПЭВМ:
\begin{itemize}
	\item машина вычислительная серии БК402;
	\item комплект периферийного оборудования, включающий в себя видеомониторы серии ВМЦ, клавиатуру, манипулятора графической информации.
\end{itemize}

ПЭВМ выполняет вычислительные функции, функции управления объектами, а также функции ввода-вывода, хранения, отображения и обработки информации.
ПЭВМ может  использоваться в  качестве:
\begin{itemize}
	\item центральной ЭВМ;
	\item автоматизированного рабочего места, при подключении видеомонитора, клавиатуры, манипулятора графической информации.
\end{itemize}

Особенности и преимущества ПЭВМ:
\begin{itemize}
	\item наличие встроенного коммутатора на 4 порта, позволяющего применять данную модификацию для построения
		локальных вычислительных сетей без применения внешнего коммутатора;
	\item наличие элементов крепления для установки их в подвижном объекте;
	\item наличие в корпусе амортизационной платформы и внутренних амортизаторов, обеспечивающих  выполнение требований к механическим воздействиям.
\end{itemize}

\subsubsection{Бесплатформенная инерциальная навигационная система}
\label{sub:lit_review:ins:bins}
БИНС -- это бесплатформенная инерциальная навигационная система на волоконно-оптических гироскопах.
БИНС -- предназначена для определения параметров движения, угловой ориентации и параметров движения наземных
транспортных средств~\cite{bins}.

Навигационная система построена на базе кварцевых акселерометров и волоконно-оптических гироскопов. Точность применяемых чувствительных элементов позволяет БИНС-Тек работать как в режиме коррекции от спутниковой навигационной системы, так и в автономном инерциальном режиме с коррекцией от одометра.

БИНС не содержит металлических частей, подверженный усталостному напряжению (карданов подвес и т.п.) и не требует технического обслуживания в процессе эксплуатации.

Режимы определения навигационных параметров:
\begin{itemize}
	\item инерциальная навигационная система;
	\item спутниковая навигационная система.
\end{itemize}
\begin{figure}[ht]
	\centering
	\includegraphics[scale=4.5]{bins}
	\caption{Бесплатформенная инерциальная навигационная система БИНС-Тек~\cite{bins}}
	\label{fig:lit_reiview:ins:bins}
\end{figure}

\subsection{Многофункциональная программно-определяемая радиостанция}
\label{sub:lit_review:radio}

Радиостанции предназначены для обеспечения передачи открытой и защищенной информации (речевых сообщений и данных) с
повышенной помехозащищенностью и скрытностью~\cite{prc9661}.
В каждой машине КМУ установлено несколько радиостанций различных типов.

Применение радиостанций:
\begin{itemize}
	\item тактическое звено управления вооруженных сил;
	\item использование в танках, БМП, БТР, автомобилях;
	\item оснащение воинских подразделений наблюдения и разведки, должностных лиц уровней командования батальонами
		(дивизионами), ротами (батареями) и взводами;
	\item оснащение командных пунктов вооруженных сил пунктов управления и узлов связи рот, батальонов и бригад.
\end{itemize}

\begin{figure}[htb]
	\centering
	\includegraphics[scale=0.35]{radio_station}
	\caption{Радиостанция серии <<Р-181>> <<Рапсодия>>~\cite{prc9661}}
	\label{fig:lit_reiview:meteo:radio_station}
\end{figure}

\subsubsection{Блок спутниковый навигационный}
\label{sub:lit_review:ins:bsn}
Блок спутниковый навигационный с цифровым датчиком атмосферного давления предназначен для приема сигналов ГЛОНАСС/GPS, включает защищенную антенну и малогабаритный приемник-измеритель.
\begin{figure}
	\centering
	\includegraphics[scale=0.55]{bsn}
	\caption{Блок спутниковый навигационный~\cite{bsn}}
	\label{fig:lit_reiview:ins:bsn}
\end{figure}

\subsubsection{Блок управления и обработки сигналов}
\label{sub:lit_review:ins:bos}
Блок управления и обработки сигналов предназначен для обработки сигналов принимаемых от бесплатформенной
навигационной системы, блока спутникового навигационного, цифрового датчика пути, их
формирования и передачи на бортовую ЭВМ.

\subsection{Автоматическая метеостанция}
\label{sub:lit_review:meteo}

Автоматическая метеостанция(АМС) это комплексный универсальный метеорологический модуль - компактное и легкое
устройство, оснащенное набором датчиков, необходимых для измерения основных метеорологических величин ~\cite{wxt530}:
\begin{itemize}
	\item направления и скорости ветра;
	\item атмосферного давления;
	\item температуры и относительной влажности.
\end{itemize}

АМС с успехом применяется в ракетных войсках и артиллерии для определения метеорологических условий стрельбы, расчет суммарных поправок на отклонение условий стрельбы.
Модуль легко устанавливается на штатной штанге командно-штабной машине дивизиона с помощью одного винта.
Поскольку модуль не имеет движущихся частей, он надежен в эксплуатации и практически не требует обслуживания.
Используемые материалы обладают высокой устойчивостью к различным загрязнения и суровым погодным условиям.
Модуль соединяется с приемным устройством двунаправленной линией передачи.

\begin{figure}
	\centering
	\includegraphics[scale=0.30]{meteo_station}
	\caption{Автоматическая метеостанция WXT530~\cite{wxt530}}
	\label{fig:lit_reiview:meteo:meteo_station}
\end{figure}

\subsection{Специальный принтер}
\label{sub:lit_review:spec_printer}
Специальный принтер предназначен для эксплуатации в составе мобильных вычислительных комплексов.
Может эксплуатироваться в процессе движения транспортных средств ~\cite{mp2200}.

Наиболее удобным для эксплуатации в машине КМУ является термопринтер. Термопринтеры гораздо более устойчивы к вибрациям и ударам, чем лазерные, матричные и струйные принтеры.

Корпус выполнен из металла, что обеспечивает стойкость к внешним механическим воздействиям.
Разъемы питания и интерфейсов (USB/LPT)\break принтера –высоконадежные «военные» байонетные металлические с металлической защитной заглушкой.
Интерфейс – высокоскоростной параллельный ECP/EPP и/или USB.

\begin{figure}
	\centering
	\includegraphics[scale=1.5]{printer}
	\caption{Специальный принтер МП2200~\cite{mp2200}}
	\label{fig:lit_reiview:spec_printer:printer}
\end{figure}

\subsection{Прибор-дальномер «Капонир»}
\label{sub:lit_review:kaponir}
Переносной телевизионно-тепловизионный наблюдательный прибор-дальномер «Капонир» предназначен для круглосуточного
ведения разведки противника и местности, определения координат своего местоположения, измерения дальности до объекта
(цели) и автоматического определения координат объекта (цели)~\cite{kaponir}.

Прибор имеет оптический, телевизионный и тепловизионный каналы наблюдения. Телевизионный и тепловизионный каналы позволяют
вести наблюдение днем и ночью. Прибор  крепится на треноге, а также  позволяет ведение наблюдения с руки.

Прибор оснащен:
\begin{itemize}
	\item приемным телевизионным каналом на фотоприемной матрице;
	\item приемным тепловизионным каналом на неохлаждаемой микроболометрической матрице;
	\item каналом лазерного дальномера;
	\item модулем электронного компаса - инклинометра;
	\item системой позиционирования GPS, ГЛОНАСС;
	\item системой расчета и определения координат объекта (цели).
	\item окуляром с OLED-микродисплеем (дисплеем на основе плоскостных органических светодиодов).
\end{itemize}

\begin{figure}
	\centering
	\includegraphics[scale=0.7]{kaponir_img}
	\caption{Прибор-дальномер <<Капонир>>~\cite{kaponir}}
	\label{fig:lit_reiview:kaponir:kaponir_img}
\end{figure}

\subsection{Лазерный целеуказатель-дальномер}
Лазерный целеуказатель-дальномер 1Д22 предназначен для разведки целей и обслуживания стрельбы наземной и корабельной
артиллерии обычными и высокоточными артиллерийскими боеприпасами с полуактивной лазерной системой наведения, а
также для обеспечения применения высокоточных боеприпасов при подсветке лазерным излучением неподвижных и движущихся объектов
вооружения и военной техники, инженерных сооружений, для корректировки артиллерийского огня с выносных наземных наблюдательных
пунктов или из машин управления огнем~\cite{lcd}.
\label{sub:lit_review:lcd}
\begin{figure}[h]
	\centering
	\includegraphics[scale=1.5]{lcd_img}
	\caption{Лазерный целеуказатель-дальномер 1Д22~\cite{lcd}}
	\label{fig:lit_reiview:lcd:lcd_img}
\end{figure}

Прибор ЛЦД обеспечивает: 
\begin{itemize}
	\item ведение разведки противника и местности;
	\item измерение горизонтальных и вертикальных углов, (имеется селекция целей и система стробирования);
	\item измерение дальности до целей (объектов), разрывов снарядов (мин);
	\item осуществление целеуказания (лазерный подсвет цели) при наведении управляемых артиллерийских боеприпасов.
\end{itemize}
