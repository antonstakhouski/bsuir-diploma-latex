\sectioncentered*{Заключение}
\addcontentsline{toc}{section}{Заключение}

В ходе дипломного проектирования было изучено множество литературных источников, относящихся к предметной области. Было
проведено исследование наиболее близких к системе аналогов и обзор технических\break средств, тестирование которых необходимо
реализовать. В результате исследования аналогов прямых аналогов разрабатываемой системы не было выявлено.

В процессе разработки были детально изучены протоколы взаимодействия устройств с ЭВМ, также были исследованы особенности
работы и применения данных устройств. Полученная информация выявила необходимость подключения как устройств,
подключенных посредством ЛВС, так и\break устройств, подключенных непосредственно к портам ЭВМ через COM порты.

В ходе анализа технических средств, тестирование которых необходимо реализовать, в архитектуре приложения было
принято решение использовать модульных подход. Выбор данного подхода обусловлен тем, что подключаемые к ЭВМ
устройства не зависят друг от друга. Данный подход позволил создать масштабируемую архитектуру, обеспечивающую простую
интеграцию новых модулей в систему.

В процессе работы над проектом были разработаны и описаны следующие схемы:
\begin{itemize}
	\item структурная схема;
	\item диаграмма последовательности;
	\item схема программы, описывающая процесс тестирования устройств;
	\item схема программы, описывающая процесс тестирования БИНС.
\end{itemize}

В процессе разработки была реализована программы настройки и тестирования устройств..
Также была реализована программа для работы с журналом тестирования. Было приведено технико-экономическое обоснование
целесообразности производства системы.

В результате была разработана система, обладающая следующими возможностями:
\begin{itemize}
	\item тестирование подключенных устройств;
	\item журналирование;
	\item настройка технических средств.
\end{itemize}

Данная система обладает следующими достоинствами:
\begin{itemize}
	\item масштабируемость;
	\item простота;
	\item многофункциональность.
\end{itemize}

Разработанный проект может быть использован для интеграции в системы автоматизации КМУ артиллерийского дивизиона.
